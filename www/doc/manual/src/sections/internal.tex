  \section{Internals}\label{sec:internal}
  \subsection{What are "service contracts"?}\label{sec:internal:contracts}
\emph{Service contracts}, as implemented by the acs-service-contract package of the OpenACS core, introduces \emph{call abstraction}, a concept with as many dimensions as actual areas of usage, to OpenACS as framework. The idea is not to reproduce existing documentation on service contracts, but to generalise the problem tackled by contracts. Service contracts aim at providing some sort of re-usability and a generic extension mechanism at the framework level. Call abstraction is a generic concept that can be found at the level of programming languages (of various flavours, OO and non-OO, functional etc.) and as infrastructure facilities in frameworks of various kind. The general motivation for call abstraction is a twofold: First, we want to assemble complex code blocks (i.e. a proc, involving sequences of calls to other code blocks) at design time (i.e. the time we conceptualise and write our program). Second, however, we do not want to specify a concrete addressee of some calls in our code block at run time. Call abstraction might be referred to as \emph{identity transparency} which simply means the actual addressees or callees are not determined at design time! The simple motivation for these two requirements is that we want to allow varying behaviour within a more generally designed picture. Varying behaviour should be able to be introduced in a pre-defined and standardised manner. Let's take an example, taken from the OO-verse of XOTcl which gives a nice show case example, easier graspable than a general description of service contracts as such. 
%
\lstset{breaklines=true,numbers=left,basicstyle=\footnotesize,frame=single}
\lstinputlisting[firstnumber=1,firstline=4,lastline=13]{../examples/xorb/example-01-call-abstraction.xotcl}
%
Imagine, you set out to provide an full-text search in your application. In OpenACS, you might think of the "search" package as a direct example. An important role in this infrastructure module is played by an indexer that either on-demand or at regular intervals triggers the indexation of new text items. The indexer is represented by an XOTcl object  "Indexer" that offers a method "index" responsible for the actual indexing walk-through. The fundamental design problem now is how to provide for the possibility to add new kinds of indexable items at an arbitrary point in time after the design and implementation of the indexer as such. To make it short, how to provide extensibility to the indexer? Key to an appropriate solution is the definition of a generic interface which needs to implemented by all indexable items that might be added in the future. 
%
\lstinputlisting[firstnumber=last,firstline=15,lastline=16]{../examples/xorb/example-01-call-abstraction.xotcl}
%
This "caller interface", represented by the XOTcl class "Indexable" shown above, stipulates an abstract call "getContent" that needs to be resolved to a concrete call at runtime. Resolving means, in an OO setting, that implementing sub-classes of Indexable come with a non-abstract method "getContent". At \begin{math}t_0\end{math}, i.e. design and implementation time of the infrastructure module, the per-object method "index" therefore exclusively refers to an abstract call "getContent". This is also known as template method, a prominent design pattern documented by \cite{gof:1994}.
%
\lstinputlisting[firstnumber=last,firstline=18,lastline=26]{../examples/xorb/example-01-call-abstraction.xotcl}
%
As can be seen from the above snippet, by implementing "Indexable" and providing a concrete method "getContent" forum entries will be indexed upon future runs of the indexer, i.e. calls to "index". Registration, in our example, is realised by establishing a sub class relationship to Indexable. If, at \begin{math}t_1\end{math}, wiki pages should be also be included in the full-text index, another specialised sub class of Indexable is needed:
%
\lstinputlisting[firstnumber=last,firstline=37,lastline=44]{../examples/xorb/example-01-call-abstraction.xotcl}
%
This is a simple sketch of what extensibility is in this reading. Note,  call abstractions in various settings come with different connotations (i.e. semantics), but they share basic ideas. Coming back to "service contracts" that realise call abstractions at a non-OO framework level, you might think of "service contracts" as abstract classes such as Indexable in the above example and of "service implementations" as registrars, represented by creating a sub class of Indexable. A more generic picture of call abstraction is given by Figure \ref{fig:advanced:ca:1}: 
  \begin{figure}[htbp]
\begin{center}
\includegraphics[width=0.55\textwidth]{img/call-abstraction-scheme.png}
\caption{Call abstraction}
\label{fig:advanced:ca:1}
\end{center}
\end{figure}
The XOTcl example (and OpenACS service contracts) would translate into the following roles shown by the schematic outline: The class Indexable (a service contract such as FtsContentProvider) stipulates takes the role of \emph{interface (descriptions)}. In this role, they have two responsibilites, first interfacing to the caller (caller interface) and interfacing to the callee (callee interface). The caller is an arbitrary code block that issues the index call on objects of type Indexable. The callee, or servant, is the concrete getContent method of any sub class derived from class Indexable (e.g. XowikiPage, or, the service implementation content\_revision). The proxy role is taken by the getContent method defined on Indexable and declared abstract (in OACS service contracts, these are operations defined on contracts).

A final clarification is needed at this point: What has become known as "remote procedure calls" (RPC) or "remote method invocations" (RMI) is the basic idea of call abstraction, but identity transparency is accompanied by \emph{location transparency}.
