

% ------------------------------------------------------------------------
% Latex - Einstellungen ***************************************************
% ------------------------------------------------------------------------
% von: Stefan Sobernig
%%%%%%%%%%%%%%%%%%%%%%%%%%%%%%%%%%%%%%%%%%%%%%%%%%%%%%%%%%%%%%%%%%%%%%%%%%
%
% ========================================================================

%% Dokumenten Klasse ===========================================================================
%
% --[ KOMA ] -----------------------------------------------------------------------------------
% Klassen: scrartcl, scrreprt, scrbook, scrletter
%  
\documentclass[%
 10pt,%           Schriftgröße
 a4paper,%	      Papier
 %DIV12,%           Seitengröße (siehe Koma Skript Dokumentation !)
 %BCOR5mm,%        Zusätzlicher Rand auf der Innenseite 
 english,%		      Sprache
 %twoside,%	      Seitenränder werden an doppelseitig angepasst
 %fleqn,%          Formeln werden linksbündig (und nicht zentriert) angezeigt 
 %openleft,%       (openright) Ein neues Kapitel beginnt immer auf einer rechten/linken Seite
 %titlepage,%      Titel wird in einer 'titlepage' Umgebung gesetzt
 %bigheadings,%    Große Überschriften (normal, small- headings)
 %halfparskip-%   Absatz wird nicht eingerückt, dafür aber um eine halbe Zeile nach unten gerückt
 %liststotoc,%     Tabellen & Abbildungsverzeichnis ins Inhaltsverzeichnis
 %bibtotoc,%       LitVerzeichnis ins Inhaltsverzeichnis
 %tablecaptionsbelow, %	Caption wird bei table unter die Tabelle gesetzt
 %headinclude,
 %nochapterprefix% Kein 'Kapitel' vor einem neuen Kapitel
 ]{scrartcl}%     Klassen: scrartcl, scrreprt, scrbook, scrletter 
% ----------------------------------------------------------------------------------------------

%% Alles was vor alles andere gehört.... =======================================================
%
%
\usepackage{calc} % Zum Rechnen innerhalb von Latex
%
% ----------------------------------------------------------------------------------------------

%% Schriften ===================================================================================
%
% -- Andere Schriftpakete -----
%
% - Times, Helvetica, Courier (Word Standard...)
%\usepackage{mathptmx}
%\usepackage[scaled=.90]{helvet}
%\usepackage{courier} 
% statt courier man kann auch die Standardschrift cmtt beibehalten
% -------------------
%
% - Palantino , Helvetica, Courier 
%\usepackage{mathpazo}
\usepackage[scaled=.92]{helvet}
%\usepackage{courier} 
% -------------------

% Weitere Times Schriften
%\usepackage{utopia}
%\usepackage{charter}
% Serifenlose Schrift
%\usepackage{avant}
% -----------------------------
%
% -- Koma Schriften --
%\renewcommand*{\headfont}{\small\sffamily\slshape}        % Kopfzeile
%\renewcommand*{\pnumfont}{\bfseries\sffamily}             % Seitenzahl
%\setkomafont{chapter}{\sffamily\Huge}                     % Chapter
%\renewcommand*{\sectfont}{\sffamily} %\rmfamily\bfseries  % Titelzeilen
%\renewcommand*{\capfont}{\small}                          % Schrift für Caption
%\renewcommand*{\caplabelfont}{\sffamily\bfseries\small}   % Schrift für 'Abbildung' usw.
%
% -- Schriften für Caption's von _nicht_- Koma kompatiblen Paketen
%
% Bei verwendung von Caption2 treten Komplikationen mit Koma auf. 
% Diese Korrekturren sind hier noch nicht implementiert !
%
%\usepackage{caption2}
%\renewcommand{\captionfont}{\small}
%\renewcommand{\captionlabelfont}{\sffamily\bfseries\small}
%\captionstyle{flushleft}
%
% -- Setzen der Titelzeile --
%\renewcommand*{\raggedsection}{\raggedright} % linksbündig, hängend
%\renewcommand*{\raggedsection}{} % Blocksatz (Latex-Standart)
%
% ----------------------------------------------------------------------------------------------
% *** Sprache *****************************
\usepackage[english]{babel}
\usepackage[T1]{fontenc}
\usepackage[latin1]{inputenc}
%------------------------------------------
% Bibliographie-Stil ===========================================================================
%\usepackage{natbib}
%\bibliographystyle{dinat}
%% Seitenlayout ================================================================================
%
% Layout laden um im Dokument den Befehl \layout nutzen zu können
%\usepackage[verbose]{layout} 
%
% Optischer Randausgleich (geht nur mit pdflatex, ansonsten hat es keine Auswirkung)
%\usepackage[activate=normal]{pdfcprot} 

% 1. Layout mit 'typearea'
% hier werden die Optionen aus \documentclass[] angewandt !
%\usepackage{typearea}

% 2. Alternative: Layout mit 'geometry'
%\usepackage{geometry}
%\geometry{a4paper}
%\geometry{twoside}
%\geometry{marginparwidth=85pt}
%\geometry{textheight=55\baselineskip}
%\geometry{textwidth=418pt} % Standard
%\geometry{bindingoffset=5mm}
%
% - Anzeigen des Layouts -
%\geometry{showframe}

% -- Zeilenabstand --
%\usepackage{setspace}
%\doublespace	        % 2-facher Abstand
%\onehalfspace        % 1,5-facher Abstand

% -- Kopfzeilen ---
\usepackage[automark, nouppercase]{scrpage2}
% Seite mit Headern
\pagestyle{scrheadings}
%
% löscht voreingestellte Stile
%\clearscrheadings 
%\clearscrplain
%
% Was steht wo...
%\ohead{\pagemark} % Oben außen: Seitenzahlen
%\ihead{\headmark} % Oben innen: Setzt Kapitel und Section automatisch
%\rehead{\leftmark} %Oben links: Kapitel
%\lohead{\rightmark} %Oben rechts: Section
%\cfoot[\pagemark]{} % Mitte unten: Seitenzahlen bei plain
\ofoot[]{\pagemark} % Mitte unten: Seitenzahlen bei headings
% Angezeigte Abschnitte im Header
%\automark[section]{chapter} %[rechts]{links}
%\automark[subsection]{section} %[rechts]{links}
% 
% Linien
%\setheadsepline{.4pt} % Linie unter dem Head
%setheadtopline{2pt} % Ganzoben
%\setfootbotline{.4pt} % Ganzunten
%
% Wie weit geht die Kopfzeile...
%\setheadwidth[0pt]{textwithmarginpar}
%\setheadwidth[0pt]{text} % standard
%
% Schriftformatierung der Kopfzeile
%\setkomafont{pagehead}{\normalfont\normalcolour\small}
%\setkomafont{pagenumber}{\normalfont\normalcolour\small}
% -- Fußnoten ---
%\deffootnote{1.5em}{1em}{
%\textsuperscript{\thefootnotemark}
%------------------------------------------
%
% ----------------------------------------------------------------------------------------------

%% sonstige Einstellungen ======================================================================
%
%-- Alternativer Absatz --
%\setlength{\parindent}{0pt} % Einzug = 0
% Siehe auch KOMA Skript zu den dortigen M�glichkeiten !!

% -- Setzen von Abbildungsbeschreibungen --
%\setcapindent*{1em} % siehe Komadokumentation Abschnitt 3.5
\setcapindent{1em} 
\addtokomafont{caption}{\scriptsize}
%
% *** Inhaltsverzeichnis ******************
\setcounter{secnumdepth}{2}  % Abbildungsnummerierung mit größerer Tiefe
\setcounter{tocdepth}{3}		 % Inhaltsverzeichnis mit größerer Tiefe
%------------------------------------------
%
% -- Ändern von Titeln --
%\usepackage{titlesec}
%\titlelabel{\thetitle.\quad} % Einen Punkt an alle Nummerierungen anfügen
%
% ----------------------------------------------------------------------------------------------

%% Bilder und Graphiken ======================================================================

% *** Bilder ******************************
% Grundlegendes
\usepackage[final]{graphicx}  % enth�lt \includegraphics !!
%\usepackage[draft]{graphicx} % Option: 'draft' für das _nicht_ Einbinden von Bildern
%\usepackage{subfigure}        % Bilder nebeneinander 
%
% Erweiterungen
%\usepackage{picinpar}         % Befehl: 'window' zum Setzen von Bildern neben Text
%
\usepackage{float}            % Stellt die Option [H] für Floats zur Verf�gung 
%\usepackage[rflt]{floatflt}   % Stellt Befehlt 'floatingfigure' zur Verf�gung
                              % [rflt] - Standard float auf der rechten Seite
%-------------------------------------------
%
% - Color Paket für farbigen Text --> _VOR_ pstricks/pst-plot !! ---
\usepackage{color} 
% Beispiel für neue Farbe: \definecolor{meingrau}{gray}{0.90}
% ------------------------------------------------------------------

% *** Graphen ******************************
\usepackage{pst-plot}
%\usepackage{psfrag}
%-------------------------------------------
%
% ----------------------------------------------------------------------------------------------


%% Mathe =======================================================================================

% *** Mathematik **************************************
%
% Neue Mathebefehle 
%\usepackage{amsmath} % der QUASI-STANDARD 
% den Befehl: eqnarray hiernach _nicht_ mehr verwenden
%------------------------------------------------------

% -- Symbole ---------
%\usepackage{amssymb}
\usepackage{latexsym} 
%\usepackage{wasysym} % (für \varangle)
%---------------------
%
% ----------------------------------------------------------------------------------------------

%% sonstige Pakete =============================================================================
%
\usepackage{makeidx}		% Index
%\usepackage{minitoc}    % Mini TOC's vor jedem Kapitel
%
\usepackage{enumerate}  % Optionen [a)], [i)] usw. 
%
\usepackage{array}
\usepackage{tabularx}   % Erweiterte Tabellen Optionen
\usepackage{multirow} 
%
%\usepackage{fancybox}   % für shadowbox, ovalbox
\usepackage{ulem}       % Zum Unterstreichen (z.B. Tensor)
\usepackage{textcomp}
\renewcommand{\familydefault}{\sfdefault}  % Symbole - z.B. \copyright
\usepackage{xspace}
% ----------------------------------------------------------------------------------------------

%% Eigene Definitionen =========================================================================
%

% tex4ht settings
%\usepackage[xhtml]{tex4ht} 
\usepackage{hyperref} % generates in-document hyper references from \ref and \label
\definecolor{darkblue}{rgb}{0,0,.5}
\hypersetup{pdftex=true, colorlinks=true, breaklinks=true, linkcolor=darkblue, menucolor=darkblue, pagecolor=darkblue, urlcolor=darkblue, citecolor=darkblue}
\usepackage{listings} % allows for including code listings similar to \verbatim and \verb


%
% ----------------------------------------------------------------------------------------------

% -- Auszuführende Befehle (vor Dokument Beginn) ----
%\makeindex
%\dominitoc
% ---------------------------------------------------

\usepackage[numbers]{natbib}
\bibliographystyle{plainnat}

% support for embedded xmp
\usepackage{xmpincl}

% display svn info
\usepackage[nofancy]{svninfo}
\cfoot[]{\svnInfoFile, rev. \svnInfoRevision, \svnInfoDate}


\newenvironment{hints}{\begin{itemize}\renewcommand\labelitemi{$\hookrightarrow$}\footnotesize}{\end{itemize}}
%\newcommand{\apilink}[3]{\href{{http://localhost:8000/api-doc/proc-view?proc=}#1+#2+#3}{#1\rightarrow#3}}
\newcommand{\site}{http://openacs-dotlrn.wu-wien.ac.at}
\newcommand{\acsapi}{\site/api-doc}
\newcommand{\xotclapi}{\site/xotcl}
\newcommand{\xotcl}{http://media.wu-wien.ac.at/langRef-xotcl.html}
\newcommand{\proclink}[3]{\href{\acsapi/proc-view?proc=#1+#2+#3}{#1\begin{math}\rightarrow\end{math}#3}}
\newcommand{\objlink}[1]{\href{\xotclapi/show-object?object=#1}{#1}}
\newcommand{\xotclref}[2]{\href{\xotcl\##1}{#2}}
\newcommand{\filelink}[2]{\href{\acsapi/procs-file-view?path=packages/#1}{#2}}










